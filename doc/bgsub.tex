\documentstyle[12pt]{article}
 \setlength{\textwidth}{16.5cm}
 \setlength{\textheight}{24cm}
 \hoffset = -1.3cm
 \voffset = -2.6cm
 \setlength{\parskip}{0.05in}
 \setlength{\parindent}{1cm}
 \setlength{\unitlength}{1mm}
% \pagestyle{plain}
 \pagestyle{myheadings}
 \markright{}

%  miscellaneous
\newcommand{\gray}{$\gamma$ ray}
\newcommand{\grays}{$\gamma$ rays}
\newcommand{\ghray}{$\gamma$-ray}
\newcommand{\ghg}{$\gamma$-$\gamma$}
\newcommand{\ghghg}{$\gamma$-$\gamma$-$\gamma$}

\begin{document}

\begin{center}
\large
{\bf       Background subtraction from in-beam HPGe coincidence data sets}
\normalsize
\end{center}
\vspace{5mm}
\begin{center}
                                D.C. Radford\\
{\em                AECL Research, Chalk River Laboratories,\\
                     Chalk River, Ontario, Canada K0J 1J0\\}
\end{center}
\vspace{8mm}
\setlength{\baselineskip}{4mm}
\begin{minipage}[t]{16.5cm}
{\bf Abstract: }
Algorithms for the subtraction of backgrounds from \ghg\ matrices and
higher-fold data sets, obtained from in-beam HPGe coincidence experiments with
heavy-ion-induced nuclear reactions, are described. The backgrounds are
parameterized as the cross-products of lower-dimensional projections of the
data and a one-dimensional background spectrum. A novel method of correcting
for a mixture of different reaction channels in the complete data set, by
making use of one or more gates on background channels in the energy region of
the ``E2 bump'', is presented. In many cases, this new method provides a
significantly better description of the background.

\end{minipage}

\newpage
\setlength{\baselineskip}{6mm}
\begin{center}
{\bf                                1. Introduction}
\end{center}

Modern HPGe detector arrays have revolutionized in-beam \ghray\ studies,
especially for high-spin nuclear structure physics with heavy-ion
fusion-evaporation reactions. In the analysis of data from such arrays, one
often creates two-dimensional histograms from double-coincidence data, or
three-dimensional histograms from triple-coincidence data, and then proceeds to
set ``gates'', {\em i.e.} specify energies for all but one of the axes and
inspect the projection onto the remaining axis. More sophisticated procedures
({\em e.g.} refs [1,2]) may attempt to perform least-squares fits directly to
the multi-fold data to extract coincident peak areas and energies. In either
case, it is usually necessary to correct the data for background counts
underlying the peaks in the gate spectrum or fitted region, arising from both
Compton-scattered \grays\ and quasicontinuum transitions.

One common method for correcting for the background underlying a gate on a peak
is to set additional gates on background regions near the peak, and subtract a
normalized fraction of the resulting projection from that obtained with the
primary gate. (In three- and higher-fold data, combinations of peak and
background gates are used.) In practise, however, it is often not possible to
find a background region free from weak contaminant peaks. This procedure also
introduces more statistical uncertainty into the results than is strictly
necessary, since the background spectrum has a similar number of counts to the
primary (peak) spectrum. In addition, this method does not remove all of the
background counts from the resulting spectrum; {\em i.e.} the result still
contains counts from Compton-scattered \grays\ and quasicontinuum transitions
in true coincidence with the peak. Therefore, when a direct least-squares
analysis of the peaks in a multi-fold data set is desired, this procedure
cannot be used, and some other method of determining the background must be
found.

A second method that is commonly used for two-dimensional data is to subtract a
fraction of the total one-dimensional projection from the gated spectrum,
normalized so that the total counts subtracted corresponds to an estimate of
the background counts in the gate. This procedure does not have the problem of
introducing additional statistical uncertainty, but still does not remove
coincident background counts from the spectrum. Generalisations of this method
to higher folds have also been made ({\em e.g.} ref. [3]).

Palameta and Waddington [4] have reported a method of subtracting the
background directly from two-dimensional \ghg\ histograms (``matrices''). The
present paper reports a prescription similar to that of ref. [4] but somewhat
simpler. The limitations of this and other prescriptions are discussed. In
particular, for nuclei populated in heavy-ion fusion-evaporation reactions,
correlations of the background with the reaction channel ({\em i.e.} different
residual nuclei) can pose a significant problem. We therefore present an
extension to the basic method which allows a correction for the different
reaction channels typically present in a data set. Extensions to three- and
higher-dimensional data sets are also given.


\begin{center}
{\bf                      2. Principles of the method}
\end{center}

In order to give a simple description of the method, we will begin by
discussing two-dimensional (2D) \ghg\ data sets. Extensions to higher folds
will be presented later.

Analysis of a \ghg\ data set will typically begin with generating a ``matrix'',
or 2D histogram of counts {\em vs.} energy {\em vs.} energy. This
matrix is then usually symmetrised, so that the two energy axes are equivalent.

Let the counts in such a data set be represented by $M_{ij}$, where $i$ and
$j$ indices are the channels corresponding to the \ghray\ energies. The
one-dimensional (1D) projection of the matrix is then
\begin{eqnarray*}
P_i  = \sum_j M_{ij} \ .
\end{eqnarray*}
Let us divide the counts in this projection into a ``background'' spectrum
$b_i$ and a ``peak'' spectrum $p_i$, such that
\begin{eqnarray*}
P_i  = b_i + p_i \ .
\end{eqnarray*}
The details of how this is done need not greatly concern us here; the
sensitivity of the resulting 2D background to this partition is addressed
below. For now, it will be sufficient to imagine that an experimenter ``draws''
the 1D background $b_i$ below the peaks in the projection in order to define
the two spectra.

The second procedure mentioned above in the introduction (subtracting part of
the 1D projection) would then correspond to subtracting the 2D background
\begin{eqnarray}
B_{ij} & = & \frac {1}{T} b_i P_j                          \nonumber \\
       & = & \frac {1}{T} (b_i b_j + b_i p_j) \ ,          \label{eq:m0} 
\end{eqnarray}
where
\begin{eqnarray}
T  =  \sum_{ij} M_{ij}  =  \sum_i P_i                      \label{eq:m0b} 
\end{eqnarray}
is the total number of counts in the matrix. This 2D background is not
symmetric, and would be used for gates set on the $i$-axis and projected onto
the $j$-axis.

As mentioned above, this does not remove the background counts in coincidence
with the peak of the gate. In order to do this, and to maintain symmetry,
we extend the procedure and use the following 2D background:
\begin{eqnarray}
B_{ij} & = & \frac {1}{T} (b_i b_j + b_i p_j + p_i b_j)    \nonumber \\
       & = & \frac {1}{T} (P_i P_j - p_i p_j) \ .          \label{eq:m1} 
\end{eqnarray}

Figure 1 illustrates the use of this background for data from the reaction
$^{124}$Sn~+~$^{30}$Si at 158 MeV, studied with the EUROGAM 1 spectrometer [5].
Shown are the total 1D projection and ``drawn'' background spectra, together
with two background-subtracted gates on a 2D matrix.
It is remarkable how well such a simple background is able to represent the
true observed background.
%especially for non-collective nuclei.
Also shown is a background-subtracted spectrum generated by setting both of the
two gates together on a 3D cube and projecting the result onto the third
dimension; this is discussed in section 5, where the extension of
equation~\ref{eq:m1} to three- and higher-folds is presented.
%The two gates shown are on fairly weak peaks; they have peak-to-background
%ratios of 0.44 and 0.58, respectively.

For the study of $E_{\gamma} - E_{\gamma}$ correlations in \ghg\ coincidences,
the COR [6] method can be used to generate a background with the same total
number of counts ($T$) as the matrix; it corresponds to the simple 2D
background
\begin{eqnarray*}
B_{ij}^{COR} = \frac {1}{T} P_i P_j \ .
\end{eqnarray*} 
No net counts remain in the matrix after this subtraction. The technique
presented here removes from such a 2D background the peak-peak part of the
projection cross-product, and can in that sense be considered a refinement of
the COR treatment.

Our method is also related to that of Palameta and Waddington [4].
In our nomenclature, their 2D background can be rewritten as:
\begin{eqnarray*}
B_{ij}^{PW} = \frac {1}{S} (b_i P^{\prime}_j + P^{\prime}_i b_j
                               - A b_i b_j) \ .
\end{eqnarray*}
Here the 1D spectrum $P^{\prime}_i$ is the projection of the matrix for the
set $K$ of all channels for which $P_j - b_j$ is zero within statistical
errors ({\em i.e.} all channels containing no significant peaks),
\begin{eqnarray*}
P^{\prime}_i = \sum_{j \in K} M_{ij} \ ,
\end{eqnarray*}
and the constants S and A are defined as
\begin{eqnarray*}
S & = & \sum_i P^{\prime}_i \ , \\
A & = & \sum_{i \in K} P^{\prime}_i / S.
\end{eqnarray*}
If we expand the set $K$ to include {\em all} channels of the $j$-axis, then we
get
\begin{eqnarray*}
P^{\prime}_i & = & P_i \ , \\
             S & = & T \ , \\
             A & = & 1 \ ,
\end{eqnarray*}
and the background of Palameta and Waddington reduces to that of
equation~\ref{eq:m1}.


\begin{center}
{\bf           3. Sensitivity of the method to the background spectrum}
\end{center}

The use of equation \ref{eq:m1} requires the definition of the background
spectrum $b_i$. One can define this spectrum simply by drawing a
piecewise-linear curve below the visible peaks in the total projection. Since
this is clearly a subjective procedure, it is important to examine the
sensitivity of the final result to this 1D background spectrum.

A simple way of doing this is to define an alternative background spectrum,
$b^{\prime}_i$, for example by multiplying $b_i$ by a factor 0.9. Thus we
obtain
\begin{eqnarray}
   b^{\prime}_i & = & 0.9 b_i                                     \nonumber \\
   p^{\prime}_i & = & P_i - b^{\prime}_i                          \nonumber \\
                & = & p_i + 0.1 b_i                               \nonumber \\
B^{\prime}_{ij} & = & \frac {1}{T} (P_i P_j - p^{\prime}_i p^{\prime}_j)
                                                                  \nonumber \\
         & = & \frac {1}{T} (P_i P_j - p_i p_j - 0.1 p_i b_j
                                   - 0.1 b_i p_j + 0.01 b_i b_j)  \nonumber \\
         & = & B_{ij} - \frac {0.1}{T} (p_i b_j + b_i p_j - 0.1 b_i b_j) \ .
                                                            \label{eq:m1r} 
\end{eqnarray}

Since the bulk of the counts are in the $P_i P_j$ term, rather than the $p_i
p_j$ term, this renormalization of $b_i$ has only a relatively minor effect on
the resulting 2D background. This can be illustrated by taking the same gates
as in figure~1, but now using this renormalized background; these spectra are
shown in figure~2. It can be seen that even a relatively large change in $b_i$
has quite a small effect, especially for the triples spectrum.

As can be seen from equation \ref{eq:m1r}, a change in $b_i$ has a very small
effect on the $b_i b_j$ term of $B_{ij}$. However, the terms in $p_i b_j$ and
$b_i p_j$ are more sensitive. Thus, for gates with very small
peak-to-background ratios, a poorly-defined 1D background will result in strong
peaks of the total projection being oversubtracted or undersubtracted.

Attempts have been made to develop techniques to automatically generate $b_i$
from $P_i$. These have not been very successful, in that the subjective
technique of removing the peaks by hand generally gives results at least as
reliable as the automatic methods.

Experience has shown that including in $b_i$ the shoulders and/or peaks
arising from $(n, n^{\prime}\gamma)$ events in the HPGe detector or surrounding
material will generally result in an improved background subtraction.
Similarly, some experiments produce X-rays in strong coincidence with
virtually all peaks of the spectrum, due to internal conversion of \grays.
In such cases, leaving the X-ray peaks in the $b_i$ spectrum will remove an
average X-ray coincident intensity from the gated spectra.


\begin{center}
{\bf             4. Correction for multiple reaction channels}
\end{center}

When nuclei are produced at high spin with heavy-ion fusion-evaporation
reactions, there are usually at least two or three strongly populated residues.
For example, in the bombardment of $^{114}$Cd with 210 MeV $^{48}$Ca, the
dominant reaction channels are $^{114}$Cd($^{48}$Ca,4n)$^{158}$Er and
$^{114}$Cd($^{48}$Ca,5n)$^{157}$Er, while weaker reaction channels include 3n
and 6n evaporation, and p$x$n evaporation.

If a residue is produced at higher spin, then more of the total energy is
carried as rotational energy, so that less of the energy brought in by the
reaction was available for the evaporation of particles. Thus, in the above
example, $^{158}$Er is generally produced with more angular momentum than
$^{157}$Er. This extra angular momentum must then be removed by \ghray\
emission subsequent to the neutron evaporation, resulting in a higher average
\ghray\ multiplicity. For collectively-rotating nuclei, most of the extra
\grays\ contribute to a quasicontinuum ``E2 bump'' in the energy region between
1 and 1.5 MeV. Consequently, the shape of this quasicontinuum \ghray\ spectrum
also differs between reaction channels. In the type of background subtraction
described here, the quasicontinuum is considered to be part of the
``background'' we are subtracting, and this will give rise to
reaction-channel-dependent effects.

This can be seen in figure~3, where spectra from gates on strong $^{156-159}$Er
\ghray\ transitions are shown. The spectra on the left have been
background-subtracted with the procedure of equation~\ref{eq:m1}. In the region
between about 1000 and 1500 keV there are serious problems with the background;
the 3n- and 4n-evaporation spectra are undersubtracted, while the 5n and 6n are
oversubtracted. Since we are subtracting an average background, the
reaction-channel dependence of the E2-bump strength is a serious problem.

The effect that gives rise to this problem, however, also points the way to a
solution. It is evident that setting a gate on background channels in the
region of the E2-bump will effectively distinguish the reaction channel of
strong peaks in the spectrum. It should then be possible to use this
information to add a correction term to the background subtraction.

This is illustrated in figure~4. The top panel shows the total 1D projection
and background from the data set of figure~3. Also shown is a wide gate on the
E2-bump region of the spectrum, from 1100 to 1600 keV, chosen to avoid strong
peaks. When this gate is applied to the 2D data, the projection spectrum shown
in the middle panel results. A 1D background can be drawn for this
E2-bump-gated spectrum in the same way as for the total projection. In the
bottom panel, the E2-bump projection has had a fraction of the total projection
subtracted, with a normalization factor chosen to give a net counts of zero in
the difference spectrum. The difference in the background spectra, with the
same normalization coefficient, is also shown. The positive (negative) peaks in
the difference spectrum correspond to transitions from the 3n and 4n (5n and
6n) evaporation residues. It is also evident that the E2-bump is more intense
in the E2-bump-gated projection than in the total projection.

Let $E$ be the set of channels included in the E2-bump gate(s), so that the
E2-bump projection spectrum is 
\begin{eqnarray*}
S_i  = \sum_{j \in E} M_{ij} \ ,
\end{eqnarray*}
with a total counts of 
\begin{eqnarray*}
C = \sum_i S_i = \sum_{i \in E} P_i \ .
\end{eqnarray*}
Let us divide the counts in the E2-bump projection into background and peaks,
\begin{eqnarray*}
S_i  = s_i + t_i \ ,
\end{eqnarray*}
and define the difference spectra of figure~4(c) as
\begin{eqnarray*}
D_i & = & S_i - \frac {C}{T} P_i \\
e_i & = & t_i - \frac {C}{T} b_i \\
d_i & = & D_i - e_i \\
    & = & s_i - \frac {C}{T} p_i \ .
\end{eqnarray*}

In order to simplify the discussion which follows, we consider
an experiment where the two primary reactions are (HI,$x$n) and
(HI,($x+1$)n). Let the 2D data, total projection and background be:
\begin{eqnarray*}
M_{ij} & = & M^a_{ij} + M^b_{ij} \\
P_{i}  & = & P^a_{i}  + P^b_{i} \\
b_{i}  & = & b^a_{i}  + b^b_{i}
\end{eqnarray*}
where the superscripts $a$ and $b$ represent spectra that would be obtained
from pure (HI,$x$n) and (HI,($x+1$)n) data sets, respectively. We define
\begin{eqnarray*}
T^n   & = & \sum_i P^n_{i} \\
S^n_i & = & \sum_{j \in E} M^n_{ij} \\
C^n   & = & \sum_i S^n_{i}
\end{eqnarray*}
where $n = a, b$. We also obtain
\begin{eqnarray*}
T   & = & T^a + T^b \\
S_i & = & S^a_i + S^b_i
\end{eqnarray*}
{\em etc.}

If the E2-bump gate on each of the individual reaction channels is
well-reproduced by the background, then the spectra $S^n_i$ can be well
represented by a fraction of the total projections $P^n_i$,
\begin{eqnarray*}
S^n_i = \frac {C^n}{T^n} P^n_i \ .
\end{eqnarray*}
We therefore obtain
\begin{eqnarray}
D_i & = & S_i - \frac {C}{T} P_i                              \nonumber \\
    & = & S^a_i + S^b_i - \frac {C}{T} (P^a_i + P^b_i)        \nonumber  \\
    & = & \frac {C^a}{T^a} P^a_i + \frac {C^b}{T^b} P^b_i -
            \frac {C^a + C^b} {T^a + T^b} (P^a_i + P^b_i)     \nonumber \\
    & = & \frac {C^a T^b - C^b T^a} {T T^a T^b}
                          (T^b P^a_i - T^a P^b_i) \\
    & = & (T^b P^a_i - T^a P^b_i) f                           \label{eq:aaa} \\
d_i & = & (T^b p^a_i - T^a p^b_i) f                           \label{eq:bbb} \\ 
e_i & = & (T^b b^a_i - T^a b^b_i) f \ ,                       \label{eq:eee}
\end{eqnarray}
where
\begin{eqnarray}
f & = & \frac {C^a T^b - C^b T^a} {T T^a T^b} \ .             \label{eq:fff}
\end{eqnarray}

For this two-reaction-channel data set, a more correct version of
equation~\ref{eq:m1} would give the 2D background
\begin{eqnarray}
B^{\prime}_{ij} =
         \frac {1}{T^a} (P^a_i P^a_j - p^a_i p^a_j) +
         \frac {1}{T^b} (P^b_i P^b_j - p^b_i p^b_j) \ .       \label{eq:ccc} 
\end{eqnarray}
With some algebra (see Appendix) we can now combine equations \ref{eq:m1}, 
\ref{eq:aaa}, \ref{eq:bbb} and \ref{eq:ccc} to get 
\begin{eqnarray}
B^{\prime}_{ij} = B_{ij} + \frac {1} {F} (D_i D_j - d_i d_j) \ ,  \label{eq:m2} 
\end{eqnarray}
where
\begin{eqnarray}
F = f^2 T T^a T^b = \frac {(C^aT^b - C^bT^a)^2} {TT^aT^b} \ .   \label{eq:FFF}
\end{eqnarray}
We can evaluate the constant $F$ by taking the sum of the difference spectrum
$D_i$ over the E2-bump gate(s) $E$,
\begin{eqnarray}
\sum_{i \in E} D_i & = & \sum_{i \in E}
          \frac {C^a T^b - C^b T^a} {T T^a T^b}
                (T^b P^a_i - T^a P^b_i)           \nonumber \\
    & = & \frac {C^a T^b - C^b T^a} {T T^a T^b}
                (T^b C^a - T^a C^b)               \nonumber \\
    & = & F \ .                                   \label{eq:m2b} 
\end{eqnarray}
Thus we are able to eliminate the need to know the individual
evaporation-residue projections in order to make use of equation~\ref{eq:m2}.

The effect of using this improved background subtraction procedure is shown on
the right-hand side of figure~3. The new method does not yield perfect results,
especially in this case for the 6n reaction channel, presumably due to the fact
that we have more than two reaction channels populated in the experiment.
Nevertheless, there is a very significant improvement, especially in the region
of the E2 bump.

When one sets gates on transitions in the energy range of the E2 bump, the
effects of the new method are still more striking. Figure~5 presents spectra
from a single gate at 1340 keV, and from a double gate at 1340 keV and the 3n
\gray\ at 208 keV, with the background subtracted using the standard procedure
((b),(d)) or the improved procedure ((a),(c)). The data set and backgrounds
used are the same as those of figures~3 and 4. The 1340 keV transition belongs
to the 4n reaction channel and not to the 3n channel, but it would be
impossible to draw that conclusion based on figure~5~(b),(d).

An alternative and perhaps more intuitive way of viewing this correction term
is to consider the difference spectrum $D_i$ as a gate on a matrix where the
background of equation~\ref{eq:m1} has already been subtracted. That is,
\begin{eqnarray}
D_i & = & \sum_{j \in E} ( M_{ij} - B_{ij} )                     \nonumber \\
    & = & S_i - \frac {1}{T} \sum_{j \in E} (P_i P_j - p_i p_j)  \nonumber \\
    & \approx & S_i - \frac {C}{T} P_i
\end{eqnarray}
since the gate $E$ is assumed to include only background channels. We then use
this projection $D_i$ and its peak and background spectra $d_i, e_i$ in the
same way as for equation~\ref{eq:m1}, except that we also need to replace the
total counts
\begin{eqnarray*}
T = \sum_i P_i
\end{eqnarray*}
with 
\begin{eqnarray*}
F = \sum_{i \in E} D_i \ .
\end{eqnarray*}
This then corrects the matrix for background correlations remaining in the E2
bump after the first background subtraction. Presumably, we could repeat this
procedure for a second problem area, using the matrix which has been
background-subtracted with equation~\ref{eq:m2} and a different gate.

It should be emphasized that the problems presented by the E2-bump are not
restricted to the background-subtraction method of equation~\ref{eq:m1}. Any
method that treats the background as a fraction of some spectrum (such as the
total 1D projection) is prone to this problem, unless that spectrum is derived
directly from background gates set close to the gate of interest. It is also
worth noting that the results presented here are not very sensitive to the
actual gate(s) used to generate the E2-bump projection. The main criterion is
simply a sensitivity to the reaction-channel dependence of the E2 bump
strength. While it is a good idea to exclude strong peaks, weak peaks such as
those in figure~4(a) can be included with no observable effects.

For residual nuclei that do not exhibit strong collective rotation at the spins
at which they are populated, there is no strong E2 bump. For these cases, the
variation of the intensity of the E2 bump with reaction channel is not as
evident, and may be absent altogether for some reactions. For example, a
comparison of the gates shown in figure~1 with those in figure~3 shows that the
effect is less pronounced for the noncollective nuclei $^{148,149}$Gd (although
a close inspection reveals that spectrum figure~1(b) is slightly oversubtracted
in the region 1.0 -- 1.6 MeV, and that of figure~1(c) slightly
undersubtracted). In cases where no reaction-channel fractionation is observed,
the use of equation~\ref{eq:m2} over that of equation~\ref{eq:m1} is not
necessary, and may in fact degrade the overall quality of the background
subtraction.


\begin{center}
{\bf                 5. Extension to triples and higher folds}
\end{center}

The new spectrometers such as GAMMASPHERE and EUROGAM gain increased
sensitivity by making use of high-fold coincidences to increase the
peak-to-background ratio for \ghray\ coincidences. Since the background is
reduced by more than an order of magnitude over that of double coincidences,
one might conclude that background subtraction of quadruple- or
quintuple-coincidence data should be less important. It must be remembered,
however, that the goal of these instruments is to make use of the increased
sensitivity to search for and examine transitions or cascades which are an
order of magnitude weaker than those observed with earlier instruments. Thus
the peak-to-background ratio at the limit of sensitivity is approximately
unchanged, and the background will have to be treated properly in order to
correctly extract information from the new high-fold data sets.

The procedures described here can be easily extended to triples and higher-fold
coincidences. Figures~ 1, 2 and 3 show results for double gates on triples data
sets, obtained with the extensions to equations~\ref{eq:m1} and \ref{eq:m2}
that are given below. Extensions of the Palameta-Waddington technique to
triples and higher folds are derived in reference [7]. Crowell {\em et al.} [3]
have also developed a high-fold background-subtraction algorithm which is in
some respects similar to that of equations~\ref{eq:m1t} and \ref{eq:m1q} below,
except that they replace the spectrum $b_i$ with estimates of the background
levels derived directly from the high-fold data in the region of the gate. They
also do not subtract the background counts in coincidence with the (n-1)-fold
peak portion of the gate on n-fold data.

The extension of equation~\ref{eq:m1} is straightforward. We consider an
n-fold cross-product of the 1D projection, keeping all terms that include at
least one power of the spectrum $b$. For example,
\begin{eqnarray*}
P_i P_j P_k & = & p_i p_j p_k \\
            & + & p_i p_j b_k + p_i b_j p_k + b_i p_j p_k \\
            & + & p_i b_j b_k + b_i p_j b_k + b_i b_j p_k \\
            & + & b_i b_j b_k \ .
\end{eqnarray*}
To get the three-dimensional background $B_{ijk}$, we drop the first term,
replace the $pp$ terms with the corresponding background-subtracted 2D
projection (in order to include the proper coincidence relationships) and
divide by powers of $T$ (to get the correct dependence on the total number of
events),
\begin{eqnarray*}
B_{ijk} =  \frac {1} {T} 
  & [ & (M_{ij}-B_{ij}) b_k + (M_{ik}-B_{ik}) b_j + (M_{jk}-B_{jk}) b_i \ \ ] \\
+  \frac {1} {T^2} 
  & [ & p_i b_j b_k + b_i p_j b_k + b_i b_j p_k \\
  &   & + b_i b_j b_k \ \ ] \ .
\end{eqnarray*}
By expanding $B_{ij}$ {\em etc.} through equation~\ref{eq:m1}, we get
\begin{eqnarray}
B_{ijk} = \frac {1}{T}
 & [ & M_{ij}b_k + M_{ik}b_j + M_{jk}b_i \ \ ]     \nonumber \\
+ \frac {1}{T^2} 
 & [ & - P_i b_j b_k - b_i P_j b_k - b_i b_j P_k   \nonumber \\
 &   & + b_i b_j b_k \ \ ] \ .                     \label{eq:m1t} 
\end{eqnarray}

Similarly, for quadruple coincidences, we obtain
\begin{eqnarray}
B_{ijkl} = \frac {1} {T} 
 & [ & C_{ijk}b_l + C_{ijl}b_k + C_{ikl}b_j + C_{jlk}b_i \ \ ]  \nonumber \\
+ \frac {1} {T^2} 
 & [ & - M_{ij} b_k b_l - M_{ik} b_j b_l - M_{jk} b_i b_l
       - M_{il} b_j b_k - M_{jl} b_i b_k - M_{kl} b_i b_j \ \ ] \nonumber \\
+ \frac {1} {T^3} 
 & [ & P_i b_j b_k b_l + b_i P_j b_k b_l +
       b_i b_j P_k b_l + b_i b_j b_k P_l                        \nonumber \\
 &   & - b_i b_j b_k b_l \ \ ] \ ,                              \label{eq:m1q}
\end{eqnarray}
where $C_{ijk}$ is the three-dimensional (3D) projection of the 4-dimensional
(4D) data set.
For completeness, we rewrite equation~\ref{eq:m1} in the same form as
equations~\ref{eq:m1t} and \ref{eq:m1q},
\begin{eqnarray*}
B_{ij} = \frac {1} {T}
  & [ & P_i b_j + b_i P_j \\
  &   & - b_i b_j \ \ ] \ .
\end{eqnarray*}

We see that in order to calculate the background, we require all of the
projections of order less than the coincidence fold, in addition to the 1D
background spectrum. The same is true for the extension of
equation~\ref{eq:m2}, except that in that method we also require all the
E2-bump-gated projections and the corresponding 1D background spectrum.

Using the format of equation~\ref{eq:m1t}, the triples version of
equation~\ref{eq:m2} reduces to
\begin{eqnarray}
B^{\prime}_{ijk} = B_{ijk} + \frac {1}{F}
 & [ & N_{ij}e_k + N_{ik}e_j + N_{jk}e_i \ \ ]     \nonumber \\
+ \frac {1}{F^2}
 & [ & - Q_i e_j e_k - e_i Q_j e_k - e_i e_j Q_k   \nonumber \\
 &   & +  \frac {U}{F} e_i e_j e_k \ \ ] \ .       \label{eq:m2t} 
\end{eqnarray}
where
\begin{eqnarray}
N_{ij} & = & \sum_{k \in E} (C_{ijk} - B_{ijk})               \nonumber \\
       & = & \sum_{k \in E} C_{ijk} - \frac {C}{T} M_{ij}
                       - \frac {1}{T} (D_i b_j + b_i D_j)     \\
Q_i    & = & \sum_{j \in E} N_{ij}                            \nonumber \\
       & = & \sum_{j \in E} \sum_{k \in E}  C_{ijk}
                   - \frac {1}{T} (C S_i + C D_i + F b_j)     \nonumber \\
       & = & \sum_{j \in E} \sum_{k \in E}  C_{ijk}
       - \frac {1}{T} (2 C D_i + \frac {C^2}{T} P_i + F b_j)  \\
U      & = & \sum_{i \in E} Q_i                               \nonumber \\
       & = & \sum_{i \in E} \sum_{j \in E}
             \sum_{k \in E}  C_{ijk} - \frac {1}{T} (3CF + \frac {C^3}{T})
\end{eqnarray}
are the E2-bump-gated projections of the background-subtracted cube.
Using the arguments of the previous section and algebra similar to that of the
Appendix, one can show that the correct 3D background for our data set with two
reaction channels reduces to equation \ref{eq:m2t}.

Similarly, the quadruples version of equation~\ref{eq:m2} is
\begin{eqnarray}
B^{\prime}_{ijkl} = B_{ijkl}
+ \frac {1} {F} 
 & [ & G_{ijk}e_l + G_{ijl}e_k + G_{ikl}e_j + G_{jlk}e_i \ \ ]  \nonumber \\
+ \frac {1} {F^2} 
 & [ & - O_{ij} e_k e_l - O_{ik} e_j e_l - O_{jk} e_i e_l
       - O_{il} e_j e_k - O_{jl} e_i e_k - O_{kl} e_i e_j \ \ ] \nonumber \\
+ \frac {1} {F^3} 
 & [ & R_i e_j e_k e_l + e_i R_j e_k e_l +
       e_i e_j R_k e_l + e_i e_j e_k R_l                        \nonumber \\
 &   & - \frac {V}{F} e_i e_j e_k e_l \ \ ] \ ,                 \label{eq:m2q}
\end{eqnarray}
where
\begin{eqnarray*}
G_{ijk} & = & \sum_{l \in E} (H_{ijkl} - B_{ijkl}) \\
N_{ij}  & = & \sum_{k \in E} G_{ijk} \\
Q_i     & = & \sum_{j \in E} N_{ij} \\
V       & = & \sum_{i \in E} Q_i
\end{eqnarray*}
are the E2-bump-gated projections of the background-subtracted 4D data set
(hypercube) $H_{ijkl}$.

Hackman and Waddington [7] have developed a very elegant notation of using
operators to describe the process of background subtraction. In our case, the
operator for the standard background subtraction is
\begin{eqnarray*}
\beta_i = 1 - \frac {b_i}{T} \sum_i
\end{eqnarray*}
for each dimension. Thus, for example,
\begin{eqnarray*}
\beta_i \beta_j M_{ij}
& = & (1 - \frac {b_i}{T} \sum_i)(1 - \frac {b_j}{T} \sum_j) M_{ij} \\
& = & M_{ij} - \frac {b_i}{T} P_j - \frac {b_j}{T} P_i
             + \frac {b_i}{T} \frac {b_j}{T} T \\
& = & M_{ij} - B_{ij} \ .
\end{eqnarray*}
Similarly,
\begin{eqnarray*}
\beta_i \beta_j \beta_k C_{ijk} = C_{ijk} - B_{ijk}
\end{eqnarray*}
{\em etc.} The operator to apply the additional E2-bump correction is
\begin{eqnarray*}
\epsilon_i = 1 - \frac {e_i}{F} \sum_i
\end{eqnarray*}
for each dimension, so that
\begin{eqnarray*}
\epsilon_i \epsilon_j \beta_i \beta_j M_{ij} & = & M_{ij} - B^{\prime}_{ij} \\
\epsilon_i \epsilon_j \epsilon_k \beta_i \beta_j \beta_k C_{ijk}
& = & C_{ijk} - B^{\prime}_{ijk}
\end{eqnarray*}
{\em etc.}

One final remark should be made about background subtraction for triples and
higher-fold data. We have assumed throughout the discussion here that the
quasicontinuum part of the background is uncorrelated, that is, that the 2D
background-background coincidences are well-described by the tensor
cross-product of the 1D background spectrum $b_i$ (with the exception of the E2
bump). While this turns out to be a reasonable approximation, it has long been
known ({\em e.g.} ref [6]) that there is often a valley along the diagonals of
the data; events where $E_{\gamma1} \approx E_{\gamma2}$ are slightly less
likely than expected from uncorrelated backgrounds. For 2D data, this problem
can be alleviated to some extent by using local backgrounds.

This valley is even deeper along the major diagonal ($E_{\gamma1} \approx
E_{\gamma2} \approx  E_{\gamma3}$) in triples data, so at first it might be
expected that the 3D background from the prescriptions described in this paper
would be poorer in this respect than that for 2D data. This, however, turns out
not to be the case. Referring to equation~\ref{eq:m1t}, one sees that the 2D
correlations for $i \approx j$, $i \approx k$ and $j \approx k$ are included in
the terms $M_{ij}b_k$, $M_{ik}b_j$ and $M_{jk}b_i$, respectively. Experience
indicates that these terms tend to cancel the background correlations in the
\ghghg\ cube.


\begin{center}
{\bf                              6. Conclusion}
\end{center}

Background-subtraction algorithms for in-beam HPGe coincidence data have been
presented. The backgrounds are parameterized as the cross-products of
lower-dimensional projections of the data and a one-dimensional background
spectrum. A novel method of correcting for a mixture of different reaction
channels in the complete data set is described. This method makes use of one or
more gates on background channels in the energy region of the ``E2 bump'', and
provides a significantly better description of the background for nuclei
exhibiting strong collective rotation at high spins.

The prescriptions for background subtraction described here are simple, fast
and reliable. The results seem to be reasonably insensitive to the particulars
of the chosen background spectra and to the gate used to generate the E2-bump
correction terms.

I thank John Simpson, Mark Riley and co-workers for the use of the EUROGAM 1
data set from the reaction of $^{114}$Cd with 210 MeV $^{48}$Ca, and G.
Hackman and J.C. Waddington for very helpful discussions.

\begin{center}
{\bf       Appendix. Algebra for the E2-bump correction in two dimensions}
\end{center}

We have (from equation \ref{eq:ccc})
\begin{eqnarray*}
B^{\prime}_{ij} & = & \frac {1}{T^a} (P^a_i P^a_j - p^a_i p^a_j) +
                      \frac {1}{T^b} (P^b_i P^b_j - p^b_i p^b_j) \\
    & = & B_{ij} + \frac {1}{T^a} P^a_i P^a_j +
                   \frac {1}{T^b} P^b_i P^b_j -
                   \frac {1}{T} P_i P_j -
                   \frac {1}{T^a} p^a_i p^a_j -
                   \frac {1}{T^b} p^b_i p^b_j) +
                   \frac {1}{T} p_i p_j \ .
\end{eqnarray*}

Collecting the terms of type $PP$ together and making use of
equations~\ref{eq:aaa}, \ref{eq:fff} and \ref{eq:FFF}, we get
\begin{eqnarray*}
     \frac {1}{T^a} P^a_i P^a_j
   + \frac {1}{T^b} P^b_i P^b_j
   - \frac {1}{T} P_i P_j
& = & \frac {1}{T^a} P^a_i P^a_j + \frac {1}{T^b} P^b_i P^b_j
    - \frac {1}{T} P^a_i P_j     - \frac {1}{T} P^b_i P_j \\
& = & \frac {1}{TT^a} P^a_i (T P^a_j - T^a P_j)
    + \frac {1}{TT^b} P^b_i (T P^b_j - T^b P_j) \\
& = & \frac {1}{TT^a} P^a_i (T^b P^a_j - T^a P^b_j)
    + \frac {1}{TT^b} P^b_i (T^a P^b_j - T^b P^a_j) \\
& = & \frac {1}{fTT^a} P^a_i D_j - \frac {1}{fTT^b} P^b_i D_j \\
& = & \frac {1}{fTT^aT^b} (T^b P^a_i - T^a P^b_i) D_j \\
& = & \frac {1}{f^2TT^aT^b} D_i D_j \\
& = & \frac {1}{F} D_i D_j \ .
\end{eqnarray*}
Similarly, collecting the terms of type $pp$ together, we have
\begin{eqnarray*}
     \frac {1}{T^a} p^a_i p^a_j
   + \frac {1}{T^b} p^b_i p^b_j
   - \frac {1}{T} p_i p_j
& = & \frac {1}{F} d_i d_j
\end{eqnarray*}
and thus
\begin{eqnarray*}
B^{\prime}_{ij} = B_{ij} + \frac {1} {F} (D_i D_j - d_i d_j) \ .
\end{eqnarray*}


\newpage
\begin{center}
{\bf                                References}
\end{center}

\begin{itemize}
\item[{[1]}] M. Bergstr\"{o}m and P. Ekstr\"{o}m, Nucl. Instr. and Meth. in
    Phys. Res. {\bf A301} (1991) 132
\item[{[2]}] D.C. Radford, Nucl. Instr. and Meth. in Phys. Res. {\bf A}
    (accompanying paper)
\item[{[3]}] B. Crowell, M.P. Carpenter, R.G. Henry, R.V.F. Janssens,
    T.L. Khoo, T. Lauritsen and D. Nisius, to be published in Nucl. Instr. and
    Meth. in Phys. Res. {\bf A}
\item[{[4]}] G. Palameta and J.C. Waddington, Nucl. Instr. and Meth. in Phys.
    Res. {\bf A234} (1985) 476
\item[{[5]}] S. Flibotte {\em et al.,} Phys. Rev. Lett {\bf 71} (1993) 688
\item[{[6]}] O. Andersen, J.D. Garrett, G.B. Hagemann, B. Herskind, D.L. Hillis
    and L.L. Riedinger, Phys. Rev. Lett. {\bf 43} (1979) 687;
             B. Herskind, Journal de Physique {\bf 41} (1980) C10-106
\item[{[7]}] G. Hackman and J.C. Waddington, to be published in Nucl. Instr.
    and Meth. in Phys. Res. {\bf A}
\end{itemize}

\setlength{\parindent}{0mm}
\newpage
\begin{center}
{\bf                                Figure Captions}
\end{center}
\vspace{5mm}

{\bf Figure 1.}
Spectra from the reaction $^{124}$Sn~+~$^{30}$Si at 158 MeV, studied with the
EUROGAM 1 spectrometer [5]. Shown are: (a) the total 1D projection and
background spectra; (b),(c) two background-subtracted gated spectra from a 2D
matrix; and (d) background-subtracted double-gated spectrum generated by
setting the two gates together on a 3D cube and projecting the result onto the
third dimension.
\vspace{5mm}

{\bf Figure 2.} 
Spectra from the same data set and identical gates as for figure~1, but using a
1D background spectrum which has been renormalized by a factor of 0.9. Shown
are: (a) the total 1D projection and renormalized background spectra; (b),(c)
background-subtracted gated spectra from a 2D matrix; and (d)
background-subtracted double-gated spectrum.
\vspace{5mm}

{\bf Figure 3.} 
Gated spectra from the reaction of $^{114}$Cd with 210 MeV $^{48}$Ca, taken
with the EUROGAM 1 spectrometer. The background has been subtracted using the
standard procedure of equation~\ref{eq:m1} (left-hand side, (a)--(f)), or using
the improved procedure of equation~\ref{eq:m2} (right-hand side, (g)--(l)).
(a),(g): Gate on the 443 keV transition in $^{158}$Er.
(b),(h): Double gate on the 443 and 522 keV transitions in $^{158}$Er.
(c),(i): Gate on the 527 keV transition in $^{157}$Er.
(d),(j): Double gate on the 527 and 622 keV transitions in $^{157}$Er.
(e),(k): Double gate on the 464 and 556 keV transitions in $^{159}$Er.
(f),(l): Double gate on the 452 and 543 keV transitions in $^{156}$Er.
\vspace{5mm}

{\bf Figure 4.} 
Illustration of the E2-bump correction for improved background subtraction.
The spectra are from the reaction of $^{114}$Cd with 210 MeV $^{48}$Ca, taken
with the EUROGAM 1 spectrometer. Shown are:
(a): Total 1D projection and background. Also shown are the limits of the gate
     set on the E2-bump region, used to produce the spectrum in (b).
(b): Gated E2-bump 1D projection and background.
(c): Normalized differences of the projections and backgrounds of parts (b) and
     (a). The normalization factor was chosen to be the ratio of the total
     counts in the projections, {\em i.e.} to give zero net counts in the
     projection difference.
\vspace{5mm}

{\bf Figure 5.} 
(a),(b): Double-gated spectra from gates at 1340 keV and 208 keV.
These transitions belong to different nuclei and are not in true coincidence.
(c),(d): Spectra from a single gate set at 1340 keV.
The background has been subtracted using the improved procedure of
equation~\ref{eq:m2} ((a),(c)), or the standard procedure of
equation~\ref{eq:m1} ((b),(d)). The data set and backgrounds are identical to
those of figures~3 and 4. Strong peaks in the spectra are labelled by the
reaction channel.

\end{document}
