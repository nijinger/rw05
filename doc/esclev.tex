\documentstyle[12pt]{article}
 \setlength{\textwidth}{16.5cm}
 \setlength{\textheight}{24cm}
 \hoffset = -1.3cm
 \voffset = -2.6cm
 \setlength{\parskip}{0.05in}
 \setlength{\parindent}{1cm}
 \setlength{\unitlength}{1mm}
% \pagestyle{plain}
 \pagestyle{myheadings}
 \markright{}

%  nuclei
\newcommand{\gd}{$^{154}$Gd}
\newcommand{\dy}{$^{156}$Dy}
\newcommand{\ho}{$^{157}$Ho}
\newcommand{\er}{$^{158}$Er}
\newcommand{\yb}{$^{160}$Yb}
\newcommand{\sn}{$^{124}$Sn}
\newcommand{\cl}{$^{37}$Cl}

%  miscellaneous
\newcommand{\gray}{$\gamma$ ray}
\newcommand{\grays}{$\gamma$ rays}
\newcommand{\ghray}{$\gamma$-ray}
\newcommand{\ghg}{$\gamma$-$\gamma$}
\newcommand{\ghghg}{$\gamma$-$\gamma$-$\gamma$}

\begin{document}

\begin{center}
\large
{\bf           ESCL8R and LEVIT8R: Software for interactive graphical
                      analysis of HPGe coincidence data sets}
\normalsize
\end{center}
\vspace{5mm}
\begin{center}
                                D.C. Radford\\
{\em                AECL Research, Chalk River Laboratories,\\
                     Chalk River, Ontario, Canada K0J 1J0\\}
\end{center}
\vspace{8mm}
\setlength{\baselineskip}{4mm}
\begin{minipage}[t]{16.5cm}
{\bf Abstract: }
Programs for analysis of \ghg\ matrices and \ghghg\ cubes from HPGe coincidence
experiments are described. The programs are intended primarily for high-spin
spectroscopy studies. Users are able to inspect background-subtracted gated
spectra, or combinations of such spectra, quickly  and easily. The programs
also make use of a proposed level scheme, provided by the user, to calculate an
expected spectrum for comparison with the observed spectrum. Electron
conversion coefficients, detection efficiency and \ghray\ energy calibrations
are included in the calculation. A graphics-based editor is included to allow
fast and easy modification of the proposed level scheme, and least-squares fits
to the matrix or cube can be performed to extract optimum values for the
energies and intensities of the level scheme transitions.

\end{minipage}

\newpage
\setlength{\baselineskip}{6mm}
\begin{center}
{\bf                          1. Introduction}
\end{center}

Modern HPGe detector arrays have revolutionized in-beam \ghray\ studies,
especially for high-spin nuclear structure physics with heavy-ion
fusion-evaporation reactions. The analysis of data from such arrays often
begins with the creation of two-dimensional histograms from double-coincidence
data, or three-dimensional histograms from triple-coincidence data. These
histograms can then be examined and by setting ``gates'', {\em i.e.} specifying
energies for all but one of the axes and inspecting the projection onto the
remaining axis. Making sense of the resulting spectra and distilling the
information from them into a deduced level scheme can be a tedious and
time-consuming process, especially for complicated level schemes with many
observed transitions.

New large \ghray\ detector arrays such as EUROGAM, GAMMASPHERE and GA.SP are
beginning to generate three-, four-, and even five-fold HPGe coincidence data
in useful quantities. The unsurpassed sensitivity of these arrays will allow
observation of nuclear levels that are well beyond the current detection
limits, not only to higher spins but also to higher excitation energies above
the yrast line, so that many new weakly-populated bands should be resolvable.
But if the information contained in these high-fold data sets is overwhelming,
so too can be the size of the data sets and the difficulty of extracting all of
the detailed information contained in them. The analysis of specific
experiments, such as the study of known superdeformed bands, should pose few
technical problems; more general analysis, however, may require substantial
enhancements to our present techniques. It may be anticipated that lack of
methods of analysing such data sets in general, powerful ways will form a major
obstacle to extracting all of the interesting physics in an expeditious and
reliable fashion.

General analysis of high-fold coincidence data to construct complete and
consistent nuclear level schemes will require sophisticated computer programs
to extract the physically interesting numbers from the raw data and present
them to the physicist in an easily assimilated manner, to keep track of all
\ghray\  assignments and expected coincidence intensities, and to quickly find
and report major discrepancies between a proposed level scheme and the data so
that they can be understood and corrected.

This paper reports on interactive graphical analysis programs which meet many
of the above criteria. These programs allow fast and easy inspection of two-
and three-dimensional data. Their main advantage, however, is that they provide
the user not only with the observed coincidence results, but also with
expectations calculated on the basis of a proposed level scheme. In this way,
it is easy for the user to find discrepancies between the proposed level scheme
and the observed results and to modify the scheme accordingly. In addition, the
book-keeping of the analysis is simplified. The level scheme is displayed on
the same screen as the observed and expected spectra, and can easily be
modified by the user. Least-squares fits directly to the 2D and/or 3D data,
with the intensities and energies of the level scheme transitions as
parameters, allow fast and accurate extraction of the important quantities
from the data. This approach considerably facilitates detailed analysis,
especially of higher-fold data sets.

\begin{center}
{\bf               2. Two-dimensional Data : Program ESCL8R}
\end{center}

ESCL8R is an interactive program for graphics-based analysis of \ghg\ data for
the deduction of level schemes. It allows fast and easy inspection of the \ghg\
matrix, making use of parameterized values of the peak shapes, peak widths,
detector efficiency and energy calibrations. It uses the two-dimensional
background-subtraction algorithm of ref. [1]. It also displays a proposed level
scheme which is modified continually as the analysis proceeds. This level
scheme is used to calculate expected results for comparison with the observed
experimental data. Least-squares fits to the matrix may be performed to extract
the optimum energies and intensities of transitions in the level scheme, with
up to 500 parameters fitted simultaneously.

\noindent
{\bf 2.1. User interface and display}

The most recent version of ESCL8R (called XmESC) makes use of X Windows and the
Motif graphical user interface. An earlier version (ESCL8R\_GLS) had only a
command-line user interface, but still used X Windows for the graphics display.
Two graphics windows are created by the program, one for the level scheme and
one for gated spectra. For users without access to X Windows, the first version
of ESCL8R displayed only the spectra and not the level scheme, and used a more
primitive level scheme file format that included no graphical information.

Figure 1 shows an X-Windows/Motif display running XmESC. A terminal-emulator
window is on the top right; this is the window that was used to launch the
application and which is used to display and ask for textual information. The
window on the top left is for the level scheme, but also includes pull-down
menus and a command widget. The bottom window displays the gated spectra. It
should be emphasized that the level-scheme and spectrum displays are for more
than just informational purposes; they are used to select or specify
transitions, bands, levels and \ghray\ energies, and thus form an important
part of the user interface.

There are several ways in which the user is able to specify gates on the
two-dimensional data, as described below. As each gate is taken, the program
automatically displays the observed and calculated spectra, the difference
between them, and the residual spectrum ({\em i.e.,} the difference spectrum
divided by the experimental uncertainty). In this way, it is easy for the user
to find places where the proposed level scheme fails to reproduce the observed
results. A graphics-based editor is incorporated to make it easy for the user
to modify or add to the level scheme.

\noindent
{\bf 2.2. Experimental data}

In addition to the coincidence matrix, the program requires a two-dimensional
(2D) background, and knowledge of the efficiency and energy calibrations and
the peak shapes and widths as a function of \ghray\ energy.

The coincidence data set used by ESCL8R is stored as a 2k x 2k-channel or 4k x
4k-channel \ghg\ ``matrix'', or 2D histogram of counts {\em vs.} energy {\em
vs.} energy. This matrix is symmetrised, so that the two energy axes are
equivalent. Future versions for the analysis of \ghg\ correlation data,
requiring non-symmetric matrices, are planned.

The program uses the background-subtraction algorithms of ref. [1], where the
2D background is defined as the sum of one-dimensional spectrum cross-products.
Rather than using a matrix with the background already removed, it subtracts
the background from each gate. This is done to allow the program to keep track
of the raw counts, and hence the uncertainties on the counts-per-channel, in
the gated spectrum.
A constant fraction of the background counts can be assigned as a systematic
uncertainty on the background subtraction.

The energy calibrations used by ESCL8R (and LEVIT8R) are simply polynomial
expressions of the \ghray\ energy as a function of channel number, usually
derived from least-squares fits to the centroids of peaks in calibration source
spectra. The program accepts polynomials of order four or less.

The parameterization of the photopeak efficiency of the detector array is more
complicated. A typical logarithmic plot of efficiency {\em vs.} \ghray\ energy
is shown in figure~2. It has the approximate appearance of two straight lines
joined by a smooth turnover. We generalize the straight lines to quadratic
functions and use the parameterization
\begin{eqnarray}
 ln (\varepsilon) = \{(A+Bx+Cx^2)^{-G} + (D+Ey+Fy^2)^{-G}\}^{-\frac{1}{G}} \ ,
\end{eqnarray}
where 
\begin{eqnarray}
x & = & ln \left ( \frac{E_\gamma}{100} \right ) \ , \\
y & = & ln \left ( \frac{E_\gamma}{1000} \right ) \ ,
\end{eqnarray}
$\varepsilon$ is the efficiency, and the \ghray\ energy $E_\gamma$ is in keV.
The parameter G determines the shape of the turn-over region between the high
and low energy efficiency curves; the larger the value of G, the sharper the
turn over at the maximum in the efficiency.
The parameter C is not usually required and is often fixed to zero. 

The program EFFIT can be used to perform least-squares fits of the above
parameterization to measured efficiency calibrations. The solid line of
figure~2 results from such a fit. The values of the parameters A through G in
this example are 6.95(13), 0.62(17), 0, 5.287(5), -0.855(13), -0.066(3) and
15(5), respectively. Since the matrix used by ESCL8R is symmetric, all
detectors are treated equivalently by the program and the efficiency parameters
used should, of course, correspond to those of the full detector array.

Two-dimensional coincident peaks are simply taken as the cross-product of
one-dimen\-sional peaks, which in turn are usually assumed to have a purely
Gaussian shape. Non-Gaussian peaks with a low-energy tail can be included, but
experience indicates that this is not usually necessary for analysis of in-beam
experiments unless the peak-to-background ratio is exceptionally good.

The widths of the peaks (FWHM) are used by the program not only in the
calculation of the expected spectrum, but also for the setting of gates. A
default gate width of one FWHM is used so that the user is able to specify a
gate by its energy alone. This considerably simplifies the inspection of the
\ghg\ matrix; for example, gates can be set by using the mouse to select
transitions in the displayed level scheme.

The FWHM of the peaks (in channels) is parameterized as
\begin{eqnarray}
 FWHM(x) = f + g \sqrt{\frac{x}{1000}} + h \frac{x}{1000} \ ,   \label{eq:fwhm}
\end{eqnarray}
where $x$ is the channel number. The first (constant) term arises from noise in
the detector and amplifier, the second term arises from the statistics of the
charge collection process, and the third (linear) term is due to
Doppler-broadening of the peaks from emission of the \grays\ from recoiling
residual nuclei. If a thick target is used, so that the recoils are stopped,
then this last term should be set to zero. Typical values of the above
parameters are approximately $f=3, \ g=1$ and $h=4$ for an energy dispersion of
0.5 keV per channel and a recoil velocity of approximately 2.5\% of the speed
of light. These parameters also depend on the detector array, electronics,
energy dispersion, {\em etc.} Their values may be modified within the program,
or fitted by least-squares regression to a selected gated spectrum, in order to
optimize the agreement between the level-scheme predictions and the observed
data.

\noindent
{\bf 2.3. Level scheme}

In order to enable a program to keep track of \ghray\ assignments and compare
the observed results with those expected from a proposed level scheme, routines
were developed to read and display level schemes stored in disk files. These
also included modified subroutines supplied by the National Nuclear Data Center
that calculate internal conversion coefficients from the Hager-Seltzer tables
[2]. Additional code was written to allow graphical editing of the files, to
generate postscript files of level scheme figures, and to check intensity
balances and energy sums. The files include data on level energies, spins and
parities, and transition energies, intensities, multipolarities, mixing ratios
and conversion coefficients, as well as on the graphical layout for display
purposes. An example of an expanded level scheme display is reproduced in
figure~3.

Editing routines allow the user to easily add, delete, modify or move bands,
levels and transitions in the level scheme. For example, a new transition
between levels that already exist in the scheme can be added simply by
indicating the initial and final levels with the mouse. The energy and
multipolarity of the transition will be taken from the differences in level
energy, spin and parity of the two levels. If one of the two levels does not
yet exist, the energy of a new \gray\ can be given by using the mouse to select
the energy of a peak in the gated spectrum, or by typing the value in keV; the
program then asks for the multipolarity, and calculates the energy, spin and
parity of the new level from the energy and multipolarity of the transition.
Both the new level and the new transition are then added to the scheme.
Simplified commands to add series of E2 or M1 transitions to the top of a band
allow very fast and easy entry of new band structures; in this case all that
the user needs to supply is the \ghray\ energies, which is usually done with
the mouse on the gated spectrum.

By incorporating these level-scheme routines, ESCL8R can calculate predicted
gated spectra to be compared with the observed spectra. That is, the program
can work backwards from the proposed level scheme, and attempt to reproduce the
observed \ghg\ matrix on the basis of the proposed intensities, branching
ratios and conversion coefficients. It can also modify the level scheme file,
and fit the \ghray\ energies and intensities directly to the 2D data, to
improve the agreement with the observed results. Consistency checks such as the
testing of energy sums and intensity balances can also be readily performed.

An additional advantage of this approach is that it allows level scheme figures
and tables of the level and transition data to be generated quickly and easily.
This also helps to document an ongoing analysis. More complicated manipulations
of the level scheme data, such as a calculation of the total population of
bands as a function of spin, {\em etc.,} are also greatly facilitated.

Since least-squares fits can be performed to the full coincidence data set, it
is important to reproduce all coincidences as correctly as possible. For
example, coincident peaks arising from the reaction channel of primary interest
may be contaminated by coincidences from other residual nuclei. Therefore a
level-scheme file to be used with an analysis program should include all bands
or transitions observed with significant intensity in the data, whether or not
they are assigned to the principal isotope being studied.

\noindent
{\bf 2.4. Setting gates}

The user is able to take and view gates on the matrix in a variety of simple
ways, for example by typing a desired energy, or by using the mouse to select
transitions in the level scheme or peaks in the spectrum. Since these specify
only a mean energy for the gate, a width is also required; by default, this is
taken as one FWHM, calculated according to equation~\ref{eq:fwhm} in subsection
2.2 above. Alternatively, limiting channels for gates can be placed by the use
of the graphics cursor or entered with the keyboard. Logical and arithmetic
combinations of gates can also be taken just as easily. Lists of gates can be
created, or read from a file, and the sum of gated spectra from such gate lists
can be calculated with a single command.

As each gate is taken, all transitions with energies that lie within the gate
limits are highlighted in the level scheme display and listed on the text
window. The program then displays the observed and calculated spectra (overlaid
in different colors), the difference between them, and the residual spectrum
({\em i.e.,} the difference spectrum divided by the experimental uncertainty);
see figure~1 for an example. In this way, it is easy for the user to find
places where the proposed level scheme fails to reproduce the observed data,
and to modify it accordingly.

If the user wishes to see where a transition in the level scheme should appear
in the spectrum, selection of that transition with the mouse in the
level-scheme display can be used to show the limits of the corresponding gate
in the spectrum window. Similarly, using the mouse to specify the energy of a
peak in the gated spectrum can highlight and list all transitions in the level
scheme that have energies within one half-width of the peak energy. A peak
search routine can be used to automatically label strong peaks in the spectrum
by their energy. 

As mentioned above, the matrix used by ESCL8R does not have the background
already removed; rather, the background is subtracted from each gate as it is
read from the matrix. In this way, the program is able to keep track of the
uncertainties on the counts-per-channel in the gated spectrum.

\noindent
{\bf 2.5. Calculation of expected \ghg\ matrix from level scheme}

In order to calculate the expected coincidence matrix (or gated spectrum), the
program first calculates the intensities of all \ghg\ coincidences. This
coincidence intensity depends on
\begin{itemize}
\item[1.] the \ghray\ intensity of the higher (feeding) transition,
\item[2.] the branching ratio from the final state of the higher transition
             through to the lower (fed) transition, and
\item[3.] the conversion coefficient of the lower transition.
\end{itemize}
Thus the program also calculates branching ratios for the transitions
deexciting each level, taking into account the electron conversion
coefficients. All \ghray\ flux entering a level is assumed to be divided
according to these branching ratios.

Once the coincident intensities have been evaluated, they can be multiplied by
the two detector efficiencies to obtain an expected coincident yield. The
\ghray\ intensities in the level scheme are in arbitrary units, so a
normalization factor is required to relate the expected yield to the observed
peak areas. This normalization coefficient is initially chosen by the program
to give the strongest observed coincidence an intensity value of about 100
units, but can be modified by the user.

The peak shapes and widths are calculated according to the paramerizations of
subsection 2.2, and the expected yield distributed accordingly. Finally, the 2D
background is added if comparison with the original matrix is required.

\noindent
{\bf 2.6. Least-squares fits}

ESCL8R provides the user with the capability to perform least-squares fits to
the observed data to extract optimum values for the \ghray\ energies and
intensities of transitions in the proposed level scheme. It is possible to fit
the energies or intensities separately, or both simultaneously. The current
version allows up to a maximum of 500 parameters to be fitted at one time, but
this limit could easily be increased. The fits are performed directly to the
full 2D coincidence matrix; any displayed gate has no effect on the results of
the fit. An extensive fit will typically have hundreds of thousands of degrees
of freedom.

The transitions to be fitted are selected by clicking with the mouse on the
level scheme display; one can choose individual transitions or all transitions
depopulating levels or bands. Choosing transitions that have already been
selected will deselect them, {\em i.e.} remove them from the list of
transitions to be fitted.

All coincidence relationships that contain information on the intensity of a
fitted transition are included in the least-squares analysis, such as
coincidences between the fitted \gray\ and the \grays\ which it feeds. If the
fitted transition is not the only one depopulating its initial level, the
program also includes coincidences between higher \grays\ and all transitions
populated from the initial state, {\em i.e.} coincidences which yield
information on the branching ratio for the fitted transition.

From the above list of quantities that determine coincidence intensities, it is
evident that any \gray\ that does not feed other \grays, and that is the
only transition from its initial level (so that it has a branching ratio of
unity independent of its intensity), has no coincidences that depend
on its intensity. Thus, for example, in an even-even nucleus the intensity of
the $2^+ \rightarrow 0^+$ transition in the ground-state band cannot be fitted
by ESCL8R, and must instead be deduced from intensity balances. If there are
two or more transitions at the bottom of a band, but none of these feed lower
transitions, then it is possible to fit all but one of them by means of their
branching ratios; however, the absolute magnitudes must then again be taken
from the intensity balance.

If a proposed level scheme is incomplete, or contains incorrectly ordered or
erroneous transitions, then the results of the least-squares fit will be
compromised. The 2D background and efficiency calibration must also be accurate
in order to obtain good results. Since the the energies of the transitions also
affect their fitted intensities, it is prudent to fit both energies and
intensities simultaneously once they are both reasonably close to their correct
values.

ESCL8R and LEVIT8R assume that the detection probability of \grays\ does not
depend on the multipolarity and/or mixing ratio of the transition. In many
cases, this is true to within a good approximation, since many detector arrays
(such as the $8\pi$ spectrometer [4]) have nearly isotropic distributions of
the HPGe detectors, and the \ghg\ matrix results from the sum of all
combinations of detectors. However, some detector systems are significantly
anisotropic. It is important for the user to remember that for data taken with
such arrays, neither the fitted intensities nor the calculated expected spectra
will be accurate.

% The errors quoted by ESCL8R and LEVIT8R for the fitted \ghray\ energies and
% intensities are statistical only, and do not include systematic errors due to
% uncertainties in the background subtraction or in  the efficiency and energy
% calibrations. As a result they should be considered to be lower limits, and
% enlarged to account for estimated systematic uncertainties.


\begin{center}
{\bf            3. Three-dimensional Data : Program LEVIT8R}
\end{center}

LEVIT8R is a program for graphical analysis of \ghghg\ data; essentially, it is
a three-dimensional version of ESCL8R. It uses the same type of parameterized
peak shape, peak width, detector efficiency and energy calibrations. It uses an
equivalent algorithm for background subtraction, extended to three dimensions
[1]. It also reads, displays and edits the same type of level scheme file. The
user interface is the same as for ESCL8R, except for extensions to allow
additional commands for gating and fitting the cube.

By incorporating the level-scheme routines, LEVIT8R calculates predicted gate
spectra to be compared with the observed spectra. That is, the program works
backwards from the proposed level scheme, and attempts to reproduce the
observed \ghghg\ cube on the basis of the proposed intensities, branching
ratios and conversion coefficients. As with ESCL8R, least-squares fits to the
data set may be performed to extract the optimum energies and/or intensities of
transitions in the level scheme, with up to 500 parameters being fitted
simultaneously. By default, the fitting is done to the two-dimensional
projection, but separate commands allow fitting of the full three-dimensional
data set. Fits to the cube consume great amounts of CPU time; they typically
have tens of millions of degrees of freedom.

One major difference from ESCL8R is that LEVIT8R allows a non-linear gain
({\em i.e.} an energy-dependent dispersion) for storage of the experimental
data. This is usually used in such a way as to yield an approximately constant
FWHM (in channels) of the peaks, independent of energy. This approach produces
a considerable saving in disk storage required for the data, especially for
high-fold data, without sacrificing usable resolution at low energies. When
such non-linear spectra are displayed, the counts-per-keV (rather than the more
conventional counts-per-channel) are shown, as a function of energy (rather
than channel number); that is, the spectra are displayed linearly with energy,
so that the displayed bin width is energy-dependent.

The use of such an energy-dependent dispersion is illustrated in figure~4, for
a typical heavy-ion fusion-evaporation reaction with a detector array such as
the $8\pi$ spectrometer 5]. In this example, a FWHM of two cube channels is
used, and the non-linear gain saves a factor of 1.9 for each dimension, over
the range 100 to 2000 keV. For the triples data, this amounts to a reduction in
the required storage space by a factor of 6.4. Even greater savings are
realized when higher energies are required; for example, if the cube covered
the range 100 to 3000 keV, using nonlinear gain saves a factor of 13 in disk
space.

A program called LUFWHM is used to generate a lookup-table, or array of
integers $L(i)$, to map (linear) ADC channels in event-by-event data to
(nonlinear) cube channel numbers; that is, the cube channel is $L$(ADC
channel). LUFWHM uses equation~\ref{eq:fwhm} to obtain the FWHM as a function
of ADC channel, and creates bins of increasing width by rounding the FWHM to
the nearest whole number of ADC channels.

A cube replay program, called INCUB8R, can then use this lookup-table mapping
to generate a cube. By ordering the three energies, the program needs to store
and increment only one-sixth of the whole cube. Two bytes of storage are
allocated to each cube channel. Such a cube typically has about 900 channels in
each dimension and requires about 240 Mbytes of disk space.

Storing only one-sixth of the cube is efficient for the replay, but has the
drawback that, in order to obtain a double-gated spectrum, a great number of
records need to be read from the cube file. In order to speed up the extraction
of gates, a second cube file format can be used, which stores one-half of the
full cube. In this second format, one byte of storage is allocated to each cube
channel, but tables of overflows ({\em i.e.,} channels that require more than
one byte since they contain more than 255 counts) are included. Such files
typically require about 370 Mbytes, and can be generated from the INCUB8R files
using a program called FOLDOUT.

Both cube formats can be read by LEVIT8R, although taking double-gates with the
second (one-half cube) format is an order of magnitude faster than with the
first format; it typically requires about 0.1s per double-gate on a modern
workstation. The user is able to take and view a gate on the two-dimensional
projection simply by, for example, typing the energy of the required gate or
clicking on transitions in the level scheme. A double-gate on the triples set
requires pairs of energies to be specified; this can again be done either with
the mouse or the keyboard. Combinations of gates can also be taken just as
easily, especially through the use of lists of gates. For example, the sum of
all pairs of gates taken from two gate lists can be generated with a single
command. The background is subtracted from each gate as it is read from the
cube, allowing the program to keep track of the uncertainties on the
counts-per-channel in the gate spectrum. 

An example of a LEVIT8R display is reproduced in figure~5.


\begin{center}
{\bf                              4. Conclusion}
\end{center}

The unsurpassed sensitivity of large \ghray\ detector arrays such as EUROGAM,
GAMMASPHERE and GA.SP will allow spectroscopic studies that go well beyond the
present frontiers of high-spin \ghray\ spectroscopy. Unfortunately, the size of
the high-fold data sets from these spectrometers, and the difficulty of
extracting all of the detailed information contained in them, can be
overwhelming.

Programs ESCL8R and LEVIT8R, for interactive graphical analysis of two-fold and
three-fold data respectively, allow fast and easy inspection of the data. They
present the user not only with the observed coincidence results, but also with
expectations calculated on the basis of a proposed level scheme. In this way,
it is easy for the user to find places where the proposed level scheme fails to
reproduce the observed results. The book-keeping of the analysis
is also simplified.

Least-squares fits directly to the 2D and/or 3D data, with the intensities and
energies of the level scheme transitions as parameters, allow expeditious
extraction of the observable quantities from the data. The level scheme can
be easily modified, and printed in the form of a postscript figure. Tables of
the level and transition data, including tests of energy sums and intensity
balances, can be generated quickly and easily. This approach considerably
improves the ease, speed and reliability of detailed data analysis, especially
in extracting ``complete'' level schemes from higher-fold data sets.

ESCL8R, LEVIT8R and associated programs run on both VMS and unix platforms.
Copies of the code are available by anonymous ftp from cu1.crl.aecl.ca.
%*** or by contacting the author by e-mail at RadfordD@crl.aecl.ca.

The assistance of R.W. MacLeod in porting software described here from VMS to
unix is gratefully acknowledged. This work has benefited greatly from helpful
discussions with many colleagues, including
R. Bark,
P. Brown,
R.W. MacLeod,
S. Pilotte,
P. Unrau,
P. Vaska,
D. Ward,
and many others.

\newpage
\begin{center}
{\bf                                References}
\end{center}

\begin{itemize}
\item[{[1]}] D.C. Radford, Nucl. Instr. and Meth. in Phys. Res. {\bf A}
    (accompanying paper)
\item[{[2]}] R.S. Hager and E.C. Seltzer, Nucl. Data Tables {\bf A4} (1968) 1
\item[{[3]}] D.C. Radford {\em et al.,} Nucl. Phys. {\bf A545} (1992) 665
\item[{[4]}] J.P. Martin {\em et al.,} Nucl. Instr. and Meth. in Phys. Res.
     {\bf A257} (1987) 301
\end{itemize}

\setlength{\parindent}{0mm}
\newpage
\begin{center}
{\bf                                Figure Captions}
\end{center}
\vspace{5mm}

{\bf Figure 1.} 
Screen dump of an ESCL8R session (Motif version).
The window on the top right is a terminal-emulator window. The window on the
top left is the level scheme and graphical user interface. The bottom window
displays spectra. Observed and predicted spectra are overlaid in the bottom of
the spectrum window; on a color screen, these are distinguishable by their
colors. The experimental spectrum shown is the sum of two gates, at 298 and 474
keV.
\vspace{5mm}

{\bf Figure 2.} 
An example of a logarithmic plot of detector efficiency {\em vs.} \ghray\
energy. The circles represent data points measured with $^{133}$Ba and
$^{152}$Eu sources, while the solid line is the result of a least-squares fit
using the program EFFIT. See text for details.
\vspace{5mm}

{\bf Figure 3.} 
An example of an expanded level scheme display. The nucleus is \ho, taken from
ref. [3].
\vspace{5mm}

{\bf Figure 4.} 
Illustration of the relation between E$_{\gamma}$, cube channel number and FWHM
when an energy-dependent dispersion is used. In this example, the energy
dispersion for the ADC is 0.5 keV/ch, and the energy dispersion in the cube is
1.0 keV/ch at low energies and 4.0 keV/ch at 2 MeV, selected to give an
approximately constant FWHM of 2 channels. In the bottom panel, the smooth
curve gives the FWHM in keV, while the sawtooth curve shows the FWHM in cube
channels. The discrete steps correspond to changes of the contraction factor
between the ADC channels and the cube channels.
\vspace{5mm}

{\bf Figure 5.} 
Screen dump of a LEVIT8R session (Motif version).
The window on the top right is a terminal-emulator window. The window on the
top left is the level scheme and graphical user interface. The bottom window
displays spectra.  Observed and predicted spectra are overlaid in the bottom of
the spectrum window; on a color screen, these are distinguishable by their
colors. The experimental spectrum shown is the sum of 90 double-gates on the
yrast superdeformed band of $^{149}$Gd.

\end{document}
